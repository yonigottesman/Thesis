\chapter{The IO Stack}
\label{chap:iostack}

\comment{Maby add a fugure with all the layers?}
%%%%%%%%%%%%%%%%%%%%%%%%%%%%%%%%%%%%%%%%%%%%%%%%%%
\section{Background}
%%%%%%%%%%%%%%%%%%%%%%%%%%%%%%%%%%%%%%%%%%%%%%%%%%
The I/O stack is the set of abstractions an I/O request passes starting at an application issuing the read/write request, and ending at the data written/read to the physical storage device. 


%%%%%%%%%%%%%%%%%%%%%%%%%%%%%%%%%%%%%%%%%%%%%%%%%%
\section{I/O Stack Components}
%%%%%%%%%%%%%%%%%%%%%%%%%%%%%%%%%%%%%%%%%%%%%%%%%%

%%%%%%%%%%%%%%%%%%%%%%%%%%%%%%%%%%%%%%%%%%%%%%%%%%
\paragraph{Filesystem}
%%%%%%%%%%%%%%%%%%%%%%%%%%%%%%%%%%%%%%%%%%%%%%%%%%
A filesystem is the methods and data structures that an operating system uses to keep track of files on a disk or partition; that is, the way files are organized on the disk.
Every filesystem implements a mapping mechanism in which file offsets are mapped to physical storage blocks.
Traditionally, UNIX-derived filesystems used per-file direct and indirect  block mapping tables to map offsets in a file to their corresponding data block. But tracking individual blocks incurs large spatial and latency
overheads when dealing with large files. Modern UNIX filesystems (e.g., ext4~\cite{mathur07ext4},
btrfs~\cite{rodeh13btrfs}, xfs~\cite{sweeney96xfs}) therefore group contiguous physical blocks into
\emph{extents} and construct extent trees, which consist of variants of B-trees~\cite{comer79btree}, to spatially
map offsets in the device to extents.
Each file is associated with an extent tree (pointed to by the file's \emph{inode}) that maps file offsets to
physical blocks.
\comment{Maybe elaborate on extents some more? moaybe a figure?}

%%%%%%%%%%%%%%%%%%%%%%%%%%%%%%%%%%%%%%%%%%%%%%%%%%
\paragraph{Block Layer}
%%%%%%%%%%%%%%%%%%%%%%%%%%%%%%%%%%%%%%%%%%%%%%%%%%
The block layer is a kernel component that handles the requests for all block devices in the system.
A request can reach this layer either from the filesystem layer, or directly from an application reading/writing straight to the raw device.
\comment{should i talk about aspects of the layer? scheduling...}


%%%%%%%%%%%%%%%%%%%%%%%%%%%%%%%%%%%%%%%%%%%%%%%%%%
\paragraph{Device Driver}
%%%%%%%%%%%%%%%%%%%%%%%%%%%%%%%%%%%%%%%%%%%%%%%%%%
The role of device drivers is to communicate with I/O devices attached to the system.
Block device drivers' get block read/write requests from the block layer

%%%%%%%%%%%%%%%%%%%%%%%%%%%%%%%%%%%%%%%%%%%%%%%%%%
\paragraph{Block Device}
%%%%%%%%%%%%%%%%%%%%%%%%%%%%%%%%%%%%%%%%%%%%%%%%%%


\comment{should i talk about new storage technologies and their performance}





